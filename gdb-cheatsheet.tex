%about: A cheatsheet for Python Debugger
%author: Florian Preinstorfer <nblock@archlinux.org>
%url: https://github.com/nblock/pdb-cheatsheet
%license: creative commons by-nc-sa
%credits:
%  - Christoph Hermes (patches)
%  - LaTeX template based on https://github.com/mtdavidson/emacs-cheatsheet
%  - http://docs.python.org/library/pdb.html
%  - https://anilattech.wordpress.com/2011/06/29/python-debugger-pdb-cheatsheet
%  - https://pythonconquerstheuniverse.wordpress.com/2009/09/10/debugging-in-python
%  - http://www.doughellmann.com/PyMOTW/pdb/
\documentclass[10pt,landscape,a4paper]{article}

\usepackage[utf8]{inputenc}
\usepackage[T1]{fontenc}
\usepackage{lmodern}
\usepackage{eqlist}
\usepackage[pdftex,hidelinks]{hyperref}
\usepackage{nopageno}
\usepackage[cm]{fullpage}
\usepackage{multicol}
\setlength{\columnsep}{1cm} % Adjust the column space here
\usepackage[hang,flushmargin]{footmisc}
\usepackage{graphicx}
\usepackage{xcolor}
\usepackage{fancyhdr}
\usepackage{tabularx}
\usepackage{enumitem}
\usepackage[top=2cm, bottom=2cm, left=1cm, right=1cm, headheight=15pt, headsep=20pt]{geometry}

%custom commands and config
\newcommand{\theauthor}{Luc Ma (\href{mailto:onion0709@gmail.com}{onion0709@gmail.com})}
\newcommand{\theversion}{1.0}
\newcommand{\keystroke}[1]{$<$#1$>$}
\newcommand{\blacksubsection}[1]{\subsection{\colorbox{black}{\makebox[\linewidth][c]{\textcolor{white}{#1}}}}}
\parindent=0cm
\parskip=1mm
\def\eqlistlabel#1{\bfseries#1}
\setcounter{secnumdepth}{-1}
\hypersetup{%
  pdftitle={Python Debugger Cheatsheet},
  pdfauthor={\theauthor},
  pdfkeywords={Python, Debugger, Cheatsheet, pdb, ipdb},
  pdfstartview={FitH}
}

\pagestyle{fancy}
\fancyhf{}
\fancyhead[C]{\large{%
    \colorbox{black}{\makebox[\linewidth][c]{\textcolor{white}{\bfseries GNU Debugger (GDB) Cheatsheet - \thepage}}}}}
\fancyfoot[C]{\small{%
    Author: \theauthor~--- version \theversion~--- license \href{https://creativecommons.org/licenses/by-nc-sa/3.0}{cc-by-nc-sa 3.0}\\
    See \url{https://github.com/lucmann/x-cheatsheet} for more information.
}}
\renewcommand{\headrulewidth}{0pt} % Remove the horizontal line at the top
\setlist[description]{style=nextline, noitemsep} % noitemsep removes whitespace between list items

\begin{document}
  \begin{multicols}{3}
    \blacksubsection{Running}
      \begin{description}
        \item[{\bfseries \$gdb \textit{<program> [core dump]}}] Start gdb (with optional core dump)
        \item[{\bfseries \$gdb --args \textit{<program> <args...>}}] Start gdb and pass arguments
        \item[{\bfseries \$gdb --pid \textit{<pid>}}] Start gdb and attach to process
        \item[set args \textit{<args...>}] Set arguments for the program
        \item[r(un)] Run the program being debugged
        \item[k(ill)] Kill the program being debugged
      \end{description}

    \blacksubsection{Breakpoints}
      \begin{description}
        \item[b(reak) \textit{<where>}] Set a new breakpoint
        \item[d(elete) \textit{<breakpoint\#>}] Remove a breakpoint
        \item[clear] Delete all breakpoints
        \item[en(able) \textit{<breakpoint\#>}] Enable a disabled breakpoint
        \item[dis(able) \textit{<breakpoint\#>}] Disable a breakpoint
      \end{description}

    \blacksubsection{Watchpoints}
      \begin{description}
        \item[w(atch) \textit{<where>}] Set a new watchpoint
        \item[delete/enable/disable \textit{<watchpoint\#>}] Like breakpoints
      \end{description}

    \blacksubsection{<where>}
      \begin{description}
        \item[\textit{function\_name}] Break/watch the named function
        \item[\textit{line\_number}] Break/watch the line number in the current file
        \item[\textit{file:line\_number}]  Break/watch the line number in the specified file 
      \end{description}

    \blacksubsection{Conditions}
      \begin{description}
        \item[break/watch \textit{<where>} if \textit{<condition>}]
              Break/watch at the given location if the condition is true. Conditions may be
              almost C expression that evaluate to true or false
        \item[condition \textit{<breakpoint\#>} \textit{<condition>}]
              Set/change the condition for an existing break- or watchpoint
        \item[]   
      \end{description}

    \blacksubsection{Movement}
      \begin{description}
        \item[s(tep)] Go to next instruction (source line), diving into function
        \item[n(ext)] Go to next instruction (source line), but don't dive into functions
        \item[\parbox{4cm}{ \vspace{0.1cm} fin(ish) \\ return \vspace{0.1cm}}] Continue until the current function returns
        \item[j(ump) \textit{<where>}] Continue program being debugged at the specified line or \textit{address}, where
                                       \textit{address} is an expression for an address to start at
        \item[c(ontinue)] Continue normal execution
        \item[c(ontinue) \textit{<count>}] Continue after skip breakpoints \textit{<count>} times
        \item[u(ntil)] continue execution until the end of a loop or until the set line
        \item[up] move one level up in the stack trace
        \item[down] move one level down in the stack trace
      \end{description}

    \blacksubsection{Examining the stack}
      \begin{description}
        \item[\parbox{4cm}{ \vspace{0.1cm} b(ack)t(race) \\ where \vspace{0.1cm}}] Show call stack
        \item[\parbox{4cm}{ \vspace{0.1cm} backtrace full \\ where full \vspace{0.1cm}}] Show call stack, also print the local variables in each frame
        \item[frame \textit{<frame\#>}] Select the stack frame to operate on
      \end{description}

    \blacksubsection{Variables}
      \begin{description}
        \item[p(rint)/format \textit{<what>}] Print content of variable/memory/register
        \item[display/format \textit{<what>}] Like "print", but print the information after each stepping instruction
        \item[undisplay \textit{<display\#>}] Remove the display with the given number
        \item[\parbox{6cm}{ \vspace{0.1cm} enable display \textit{<display\#>} \\ disable display \textit{<display\#>} \vspace{0.1cm}}]
              En- or disable the display with the given number
      \end{description}

    \blacksubsection{<what>}
      \begin{description}
        \item[expression] Almost any C expression, including function calls (must be prefixed with a cast to tell GDB the return value type)
        \item[file\_name::variable\_name] Content of the variable defined in the named file (static variables)
        \item[function::variable\_name] Content of the variable defined in the named function (if on the stack)
        \item[\textit{\{type\}}address] Content at the address, interpreted as being of C type \textit{type}
        \item[\$register] \parbox{8cm}{Content of named register. Interesing registers: \\
                                       \textit{\$pc}: program counter\\
                                       \textit{\$sp}: stack pointer\\
                                       \textit{\$fp (\$ebp)}: frame pointer\\
                                       \textit{\$eip}: instruction pointer}
      \end{description}

    \blacksubsection{Memory}
      \begin{description}
        \item[x/nfu \textit{<address>}] \parbox{8cm}{Print the memory at the given address. \\
                                        \textit{n}: How many units to print (default 1) \\
                                        \textit{f}: Format specifier (like "print") \\
                                        \textit{u}: Unit}
      \end{description}

    \blacksubsection{Format}
      \begin{description}
        \item[a] Pointer
        \item[c] Read as integer, print as character
        \item[d] Integer, signed decimal
        \item[f] Floating point number
        \item[o] Integer, octal
        \item[s] Try to treat as C string
        \item[t] Integer, binary (t = "two")
        \item[u] Integer, unsigned decimal
        \item[x] Integer, hexadecimal
      \end{description}

    \blacksubsection{Threads}
      \begin{description}
        \item[thread \textit{<thread\#>}] Choose thread to operate on
      \end{description}

    \blacksubsection{Manipulating the program}
      \begin{description}
        \item[set var \textit{<variable\_name>}=\textit{<value>}] Change the content of a variable to the given value
        \item[return \textit{<expression>}] Force the current function to return immediately, passing the given value
      \end{description}

    \blacksubsection{Sources}
      \begin{description}
        \item[dir(ectory) \textit{<directories>}] Add \textit{<directories>} to the source file search path, \textit{<directories>} can be separated by colons
        \item[\parbox{6cm}{ \vspace{0.1cm} list \\ list \textit{<filename>:<function>} \\ list \textit{<filename>:<line\_number>} \\ list \textit{<first>,<last>}\vspace{0.1cm}}]
              Show the current or given source code snippet. The \textit{<filename>} can be omitted. If \textit{<last>} is omitted, 10 lines are shown.
        \item[set listsize \textit{<count>}] Set how many lines to show in "list"
      \end{description}

    \blacksubsection{Signals}
      \begin{description}
        \item[handle \textit{<signal>} \textit{<options>}]
              \parbox{8cm}{ \vspace{0.1cm}Set how to handle signals. \textit{<options>} can be: \\
                                       \textit{(no)print}: (Don't) print a message when the signal occurs\\
                                       \textit{(no)stop}: (Don't) stop the program when the signal occurs\\
                                       \textit{(no)pass}: (Don't) pass the signal to the program}
      \end{description}

    \blacksubsection{Informations}
      \begin{description}
        \item[\parbox{6cm}{ \vspace{0.1cm} disassemble \\ disassemble \textit{<where>} \vspace{0.1cm}}]
              disassemble the current function or the given location
        \item[i(nfo) args] Print the arguments to the function of the current stack frame
        \item[i(nfo) b(reakpoints)] Print informations about the break- and watchpoints
        \item[i(nfo) display] Print informations about "displays"
        \item[i(nfo) locals] Print the local variables in the currently selected stack frame
        \item[i(nfo) sharedlibrary] List loaded shared libraries
        \item[i(nfo) signals] List all signals and how they are currently handled
        \item[i(nfo) threads] List all threads
        \item[show directories] Print all directories in which GDB searches for source files
        \item[show listsize] Print how many are shown in "list" command
        \item[whatis \textit{variable\_name}] Print type of named variable
      \end{description}

  \end{multicols}
\end{document}
