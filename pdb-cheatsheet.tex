%about: A cheatsheet for Python Debugger
%author: Florian Preinstorfer <nblock@archlinux.org>
%url: https://github.com/nblock/pdb-cheatsheet
%license: creative commons by-nc-sa
%credits:
%  - Christoph Hermes (patches)
%  - LaTeX template based on https://github.com/mtdavidson/emacs-cheatsheet
%  - http://docs.python.org/library/pdb.html
%  - https://anilattech.wordpress.com/2011/06/29/python-debugger-pdb-cheatsheet
%  - https://pythonconquerstheuniverse.wordpress.com/2009/09/10/debugging-in-python
%  - http://www.doughellmann.com/PyMOTW/pdb/
\documentclass[10pt,landscape,a4paper]{article}

\usepackage[utf8]{inputenc}
\usepackage[T1]{fontenc}
\usepackage{lmodern}
\usepackage{eqlist}
\usepackage[pdftex,hidelinks]{hyperref}
\usepackage{nopageno}
\usepackage[cm]{fullpage}
\usepackage{multicol}
\setlength{\columnsep}{1cm} % Adjust the column space here
\usepackage[hang,flushmargin]{footmisc}
\usepackage{graphicx}
\usepackage{xcolor}
\usepackage{fancyhdr}
\usepackage{tabularx}
\usepackage{enumitem}
\usepackage[top=2cm, bottom=2cm, left=1cm, right=1cm, headheight=15pt, headsep=20pt]{geometry}

%custom commands and config
\newcommand{\theauthor}{Florian Preinstorfer (\href{mailto:nblock@archlinux.us}{nblock@archlinux.us})}
\newcommand{\theversion}{1.2}
\newcommand{\keystroke}[1]{$<$#1$>$}
\newcommand{\blacksubsection}[1]{\subsection{\colorbox{black}{\makebox[\linewidth][c]{\textcolor{white}{#1}}}}}
\parindent=0cm
\parskip=1mm
\def\eqlistlabel#1{\bfseries#1}
\setcounter{secnumdepth}{-1}
\hypersetup{%
  pdftitle={Python Debugger Cheatsheet},
  pdfauthor={\theauthor},
  pdfkeywords={Python, Debugger, Cheatsheet, pdb, ipdb},
  pdfstartview={FitH}
}

\pagestyle{fancy}
\fancyhf{}
\fancyhead[C]{\large{%
    \colorbox{black}{\makebox[\linewidth][c]{\textcolor{white}{\bfseries Python Debugger Cheatsheet - \thepage}}}}}
\fancyfoot[C]{\small{%
    Author: \theauthor~--- version \theversion~--- license \href{https://creativecommons.org/licenses/by-nc-sa/3.0}{cc-by-nc-sa 3.0}\\
    See \url{https://github.com/nblock/pdb-cheatsheet} for more information.
}}
\renewcommand{\headrulewidth}{0pt} % Remove the horizontal line at the top

\setlist[description]{style=nextline, noitemsep} % noitemsep removes whitespace between list items

\begin{document}
  \begin{multicols}{3}
    \blacksubsection{Getting started}
      \begin{description}
        \item[{\bfseries import pdb;pdb.set\_trace()}]
             start pdb from within a script
        \item[{\bfseries python -m pdb \keystroke{file.py}}] start pdb from the commandline
      \end{description}

    \blacksubsection{Basics}
      \begin{description}
        \item[h(elp)] print available commands
        \item[h(elp) \textit{command}] print help about \textit{command}
        \item[q(quit)] quit debugger
      \end{description}

    \blacksubsection{Examine}
      \begin{description}
        \item[p(rint) \textit{expr}] print the value of \textit{expr}
        \item[pp \textit{expr}] pretty-print the value of \textit{expr}
        \item[w(here)] print current position (including stack trace)
        \item[l(ist)] list 11 lines of code around the current line
        \item[l(ist) \textit{first}, \textit{last}] list from \textit{first} to \textit{last} line number
        \item[a(rgs)] print the args of the current function
      \end{description}

    \blacksubsection{Miscellaneous}
      \begin{description}
        \item[!\textit{stmt}] treat \textit{stmt} as a Python statement instead of a pdb command
        \item[alias \textit{map} \textit{stmt}] map Python statement as a map command
        \item[alias \textit{map} <\textit{arg1} \ldots> \textit{stmt}] pass arguments to Python statement. \\
            \textit{stmt} includes \%1, \%2, \ldots literals.
      \end{description}
      Save pdb commands to local \keystroke{./.pdbrc} file for repetitive access.

    \blacksubsection{Movement}
      \begin{description}
        \item[\keystroke{ENTER}] repeat the last command
        \item[n(ext)] execute the current statement (step over)
        \item[s(tep)] execute and step into function
        \item[r(eturn)] continue execution until the current function returns
        \item[c(ontinue)] continue execution until a breakpoint is encountered
        \item[u(p)] move one level up in the stack trace
        \item[d(own)] move one level down in the stack trace
        \item[until] continue execution until the end of a loop or until the set line
        \item[j(ump)] set the next line that will be executed (local frame only)
      \end{description}

    \blacksubsection{Breakpoints}
      \begin{description}
        \item[b(reak)] show all breakpoints with its \textit{number}
        \item[b(reak) \textit{lineno}] set a breakpoint at \textit{lineno}
        \item[b(reak) \textit{lineno}, \textit{cond}] stop at breakpoint \textit{lineno} if Python condition \textit{cond} holds, e.g.\ i==42
        \item[b(reak) \textit{file}:\textit{lineno}] set a breakpoint in \textit{file} at \textit{lineno}
        \item[b(reak) \textit{func}] set a breakpoint at the first line of a \textit{func}
        \item[tbreak \textit{lineno}] set a temporary breakpoint at \textit{lineno}, i.e.\ is removed when first hit
        \item[disable \textit{number}] disable breakpoint \textit{number}
        \item[enable \textit{number}] enable breakpoint \textit{number}
        \item[clear \textit{number}] delete breakpoint \textit{number}
      \end{description}

  \end{multicols}
\end{document}
